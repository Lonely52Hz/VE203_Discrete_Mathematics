\newcommand{\plogo}{\fbox{$\mathcal{PL}$}} % Generic dummy publisher logo

%\usepackage[utf8]{inputenc} % Required for inputting international characters
%\usepackage[T1]{fontenc} % Output font encoding for international characters
%\usepackage{fouriernc} % Use the New Century Schoolbook font
\documentclass{article}[12pt]
\usepackage[margin=2.5cm]{geometry}
\usepackage{enumerate}
\usepackage{booktabs}
\usepackage{amsmath}
\newtheorem{theorem}{Theorem}  
\newtheorem{lemma}{Lemma}  
\usepackage{pifont}
\newtheorem{proof}{Proof}
\usepackage{caption}
\usepackage{amssymb}
\usepackage{ulem}
\usepackage{graphicx}
\usepackage{subfigure}
\usepackage{geometry}
\usepackage{multirow}
\usepackage{multicol}
\usepackage{indentfirst}
\usepackage{xcolor}
\usepackage{verbatim}
%\usepackage{ctex}
\usepackage{gauss}
\usepackage{float}
\usepackage[version=4]{mhchem}
\usepackage[linesnumbered,ruled,vlined]{algorithm2e}
%\usepackage{algorithm}
\usepackage{algorithmic}
%\renewcommand{\algorithmicrequire}{\textbf{Input:}}
%\renewcommand{\algorithmicensure}{\textbf{Output:}}
\SetKwInput{KwInput}{Input}                % Set the Input
\SetKwInput{KwOutput}{Output} 
\begin{document}
\noindent

%========================================================================
\noindent\framebox[\linewidth]{\shortstack[c]{
\Large{\textbf{VE203 Assignment 7}}}}
\begin{center}
\footnotesize{\quad Name: YIN Guoxin\quad Student ID: 517370910043}


\end{center}
%=======================================================================

\noindent \textbf{Q1.}
Since $a c \equiv b c\ (\bmod\  m)$, we have $m|(ac-bc)$. Since $\frac{m}{\operatorname{gcd}(c, m)}|m$, we also have $\frac{m}{\operatorname{gcd}(c, m)}|(ac-bc)=(a-b)c$ and also $\operatorname{gcd}(\frac{m}{\operatorname{gcd}(c, m)},c)=1$. From this, we know that $\operatorname{gcd}(c, m)|(a-b)$, which means $a \equiv b\left(\bmod\ \frac{m}{\operatorname{gcd}(c, m)}\right)$.
\\ 
\\
\noindent \textbf{Q2.}
Since 36$x \equiv 75(\bmod\ 1309)$, we have $1309|(36x-75)$, i.e. there exists $m\in \mathbb{Z}$ such that $75=36x+1209m$. Since 
\begin{align*}
1309&=36\times 36+13\\
36&=2\times 13+10\\
13&=10+3\\
10&=3\times 3+1\\
3&=3\times 1,
\end{align*} we know that gcd(1309,36)=1. Therefore, there we have 
\begin{align*}
1&=10-3\times 3\\
&=10-3\times (13-10)\\
&=4\times 10-3\times 13\\
&=4\times (36-2\times 13)-3\times 13\\
&=4\times 36-11\times 13\\
&=4\times 36-11\times (1309-36\times 36)\\
&=400\times 36-11\times 1309,
\end{align*}
which means 1=$400\times 36-11\times 1309$. Therefore $75=400\times 75\times 36-11\times 75\times 1309=30000\times 36-825\times 1309$. Hence, $x=30000$ is a solution.\\ \\
\noindent \textbf{Q3.}
To solve this system, we need to solve 
\begin{align*}
x&\equiv 2 (\bmod\ 5)\\
x&\equiv 4 (\bmod\ 7)\\
x&\equiv 4 (\bmod\ 13)
\end{align*}
Now, 
\begin{align*}
&m=m_1m_2m_3=455\\
&M_1=\frac{455}{5}=91\\
&M_2=\frac{455}{7}=65\\
&M_3=\frac{455}{13}=35\\
\end{align*}
To find $y_1$, 
\begin{align*}
91&=18\times 5+1\\
1&=91-18\times 5\\
\end{align*}
Hence, let $y_1=1$. To find $y_2$, 
\begin{align*}
65&=9\times 7+2\\
7&=3\times 2+1\\
1&=7-3\times 2\\
&=7-3\times (65-9\times 7)\\
&=28\times 7-3\times 65
\end{align*}
Hence, let $y_2=-3$. To find $y_3$, 
\begin{align*}
35&=2\times 13+9\\
13&=9+4\\
9&=2\times 4+1\\
1&=9-2\times 4\\
&=9-2\times (13-9)\\
&=3\times 9-2\times 13\\
&=3\times (35-2\times 13)-2\times 13\\
&=3\times 35-8\times 13
\end{align*}
Hence, let $y_3=3$. Therefore, 
\begin{align*}
x=\sum_{k=1}^{n}a_kM_ky_k=2\times 91\times 1+4\times 65\times (-3)+4\times 35\times 3=-178 (\bmod 455). 
\end{align*}
Therefore, the solutions are -178+455$y$, where $y\in \mathbb{Z}$.
\\

\noindent \textbf{Q4.}
\begin{align*}
6|(x-5)
\Rightarrow &3|(x-2)\\
&2|(x-1),\\
10|(x-3) \Rightarrow &5|(x-3)\\
&2|(x-1),\\
15|(x-8) \Rightarrow &5|(x-3)\\
&3|(x-2)\\
\end{align*}
Since there are no contradiction between the equations in the right hand side, there must exist solutions and we turn the system into 
\begin{align*}
x&\equiv 1 (\bmod\ 2)\\
x&\equiv 2 (\bmod\ 3)\\
x&\equiv 3 (\bmod\ 5)
\end{align*}
Now, 
\begin{align*}
&m=m_1m_2m_3=30\\
&M_1=\frac{30}{2}=15\\
&M_2=\frac{30}{3}=10\\
&M_3=\frac{30}{5}=6\\
\end{align*}
To find $y_1$, 
\begin{align*}
15&=7\times 2+1\\
1&=15-7\times 2\\
\end{align*}
Hence, let $y_1=1$. To find $y_2$, 
\begin{align*}
10&=3\times 3+1\\
1&=10-3\times 3\\
\end{align*}
Hence, let $y_2=1$. To find $y_3$, 
\begin{align*}
6&=5+1\\
1&=6-5
\end{align*}
Hence, let $y_3=1$. Therefore, 
\begin{align*}
x=\sum_{k=1}^{n}a_kM_ky_k=1\times 15\times 1+2\times 10\times 1+3\times 6\times 1=53 (\bmod 30). 
\end{align*}
Therefore, the solutions are 53+30$y$, where $y\in \mathbb{Z}$.
\\

\noindent \textbf{Q5.}
Now, 
\begin{align*}
&m=m_1m_2m_3m_4=6545\\
&M_1=\frac{6545}{5}=1309\\
&M_2=\frac{6545}{7}=935\\
&M_3=\frac{6545}{11}=595\\
&M_4=\frac{6545}{17}=385
\end{align*}
To find $y_1$, 
\begin{align*}
1309&=261\times 5+4\\
5&=4+1\\
1&=5-4\\
&=5-(1309-261\times 5)\\
&=262\times 5-1309
\end{align*}
Hence, let $y_1=-1$. To find $y_2$, 
\begin{align*}
935&=133\times 7+4\\
7&=4+3\\
4&=3+1\\
1&=4-3\\
&=4-(7-4)\\
&=2\times 4-7\\
&=2\times (935-133\times 7)-7\\
&=2\times 935-267\times 7
\end{align*}
Hence, let $y_2=2$. To find $y_3$, 
\begin{align*}
595&=11\times 54+1\\
1&=595-11\times 54
\end{align*}
Hence, let $y_3=1$. To find $y_4$, 
\begin{align*}
385&=22\times 17+11\\
17&=11+6\\
11&=6+5\\
6&=5+1\\
1&=6-5\\
&=6-(11-6)\\
&=2\times 6-11\\
&=2\times (17-11)-11\\
&=2\times 17-3\times (385-22\times 17)\\
&=68\times 17-3\times 385
\end{align*}
Hence, let $y_4=-3$. Therefore, 
\begin{align*}
x=\sum_{k=1}^{n}a_kM_ky_k=5\times 1309\times (-1)+3\times 935\times 2+8\times 595\times 1+2\times 385\times -3=1515 (\bmod\ 6545). 
\end{align*}
Therefore, the solutions are 1515+6545$y$, where $y\in \mathbb{Z}$.
\\



\noindent \textbf{Q6.}
Suppose, for a contradiction, that there exists $M,C\in \mathbb{N}$ such that for all $n>M$, $n \log _{2}(n)\leq C\log _{2}(n)$. But it cannot be true since when $n>C$, $n \log _{2}(n)> C\log _{2}(n)$ always hold.\\

\noindent \textbf{Q7.}
We just need to prove that $\log_a(n)=O\log_b(n)$ where $a,b$ are positive integers greater than 1. Then, 
\begin{align*}
\lim_{n\rightarrow \infty}\frac{\log_a(n)}{\log_b(n)}=\lim_{n\rightarrow \infty}\frac{\frac{1}{n(\ln a)}}{\frac{1}{n(\ln b)}}=\frac{\ln b}{\ln a},
\end{align*}
which means $\log_a(n)=O\log_b(n)$. When $a=2,b=10$, $a=2,b=e$, $a=10,b=2$, $a=10,b=e$, $a=e,b=2$, $a=e,b=10$, we can see that they have the same order from the equation we proved.
\\

\noindent \textbf{Q8.}
For all $x\in \mathbb{R}$, $\left\lfloor x^{3}-4\right\rfloor\leq x^3-4$. For $x\leq 2$, $|\left\lfloor x^{3}-4\right\rfloor|\leq |x^3-4|<|x^3+4|=|x^3(1+\frac{4}{x^3})|\leq |x^3(1+4)|=|(x^3)|\times 5$. Therefore, by choosing $C\in \mathbb{N}$ with $C\geq 5$, we can see that $\left\lfloor x^{3}-4\right\rfloor$ is $O(x^3)$.
Conversely, $\left\lfloor x^{3}-4\right\rfloor\geq x^3-5$. It is clear that for all $\epsilon>0$, there exists $D\in \mathbb{R}$ with $D>0$ such that for all $x>D$,
\begin{align*}
|1-\frac{5}{x^3}|\geq 1-\epsilon
\end{align*}
So, choosing $\epsilon=\frac{1}{2}$, we get $D\in \mathbb{R}$ with $D>0$ such that for all $x>D$,
\begin{center}
$|1-\frac{5}{x^3}|\geq \frac{1}{2}$ and $|\left\lfloor x^{3}-4\right\rfloor|> |x^3-5|=x^3|1-\frac{5}{x^3}|\geq \frac{1}{2}x^3$
\end{center}
So, if $C\in \mathbb{N}$ is such that $C\geq 2$, then $|x^3|\leq 2\times |\left\lfloor x^{3}-4\right\rfloor|$, which shows that $x^3$ is $O(\left\lfloor x^{3}-4\right\rfloor)$.\par
Therefore, they have the same order.\\

\noindent \textbf{Q9.}
No. For $n$ large enough, 
\begin{align*}
\lim_{n\rightarrow \infty}\frac{n^n}{n^{n-k}}=\lim_{n\rightarrow \infty}n^k\rightarrow \infty,
\end{align*}
So there doesn't exist $C\in \mathbb{R}$ with $C\geq 0$ such that $\lim_{n\rightarrow \infty}\frac{n^n}{n^{n-k}}=C$, which means $n^n\not=O(n^{n-k}$).

\newpage
\noindent \textbf{Q10.}
\begin{enumerate}[(i)]
\item Note that the summation only makes sense when $n\geq 2$, therefore, we only discuss conditions when $n\geq 2$.
\begin{itemize} 
\item When $n=2$, $\sum_{j=2}^{2} \frac{1}{j}=\frac{1}{2}$ and $\int_{1}^{2} \frac{1}{x} d x=\ln x|^2_1=\ln 2-\ln 1=\ln 2>\frac{1}{2}$.
\item If $n>2$ and $\sum_{j=2}^{n} \frac{1}{j}<\int_{1}^{n} \frac{1}{x} d x$ holds, suppose for $n+1$, $\sum_{j=2}^{n+1} \frac{1}{j}=\sum_{j=2}^{n} \frac{1}{j}+\frac{1}{n+1}<\int_{1}^{n} \frac{1}{x} d x+\frac{1}{n+1}$. Besides, $\int_{1}^{n+1} \frac{1}{x} d x=\int_{1}^{n} \frac{1}{x} d x+\int_{n}^{n+1} \frac{1}{x} d x=\int_{1}^{n} \frac{1}{x} d x+\ln x|^{n+1}_n=\int_{1}^{n} \frac{1}{x} d x+\ln (n+1) - ln(n)>\int_{1}^{n} \frac{1}{x} d x+\frac{1}{n+1}$. Therefore, we have $\sum_{j=2}^{n+1} \frac{1}{j}<\int_{1}^{n+1} \frac{1}{x} d x$.
\end{itemize}
\item $H(n)=\sum_{k=0}^{n-1} \frac{1}{n-k}=\sum_{j=1}^{n} \frac{1}{j}=1+\sum_{j=2}^{n} \frac{1}{j}<1+\int_{1}^{n} \frac{1}{x} d x<1+\ln n$. So, when $n>e$, $\ln e>1$, $1+\ln e<2\ln n$. Therefore, we have when $n>e$, $H(n)<2\ln n$, which means $O(\ln (n))$.
\end{enumerate}


\noindent \textbf{Q11.}
\begin{enumerate}[(i)]
\item The pseudocode is as follow,\\
\begin{algorithm}[H]
  \KwInput{$a_1,...,a_n$, $n$ unsorted elements}
  \KwOutput{all the $a_i$, $1\leq i\leq n$ in increasing order}
  \For {$k\rightarrow 2$ to $n$} 
  {
  $i\leftarrow 1$\\
  $j\leftarrow k-1$\\
  \While{$i<j$}
   {
   		$m\leftarrow \left\lfloor (i+j)/2\right\rfloor$;\\
   		\If{$a_k>a_m$} {$i\leftarrow m+1$;}
   		\Else {$j\leftarrow m$}
   }
  $p\leftarrow a_k$;\\
  \For{$l\leftarrow 0$ to $k-i-1$} {$a_{k-l}\leftarrow a_{k-l-1}$;}
  $a_i\leftarrow p$
  }
  \Return $(a_1,...,a_n$) in increasing order.
\end{algorithm}
\item 
\begin{itemize}
\item Insertion Sort Algorithm: 8+(1+2)+(1+3)+(4+1)+(1+5)+(4+3)+(3+5)+(2+7)=50
\item Binary Insertion Sort Algorithm: 8+(1+3)+(3+4)+(3+2)+(5+6)+(5+4)+(5+6)+(7+8)=70
\end{itemize}
\item We define a function $f:\mathbb{N}\longrightarrow\mathbb{N}$ that on input $n$ counts the worst-case number of comparisons needed sort a list of length $n$. For a specific $j^{th}$ element to sort, if there is $k$ comparisons during the\textbf{ while} loop, which means $i=k-1$, there will be $j-i-1+1+1=j-i+1=j-k+1$ comparisons during the inner \textbf{for} loop. So there are $k+j-k+1=j+1$ comparisons in total during the $j^{th}$ passes. Adding all $j$ we get
\begin{align*}
\sum_{j=2}^n (j+1)=\frac{n^2}{2}+\frac{3n}{2}-2
\end{align*}
Besides, we also have $n$ comparisons for the outer for loops, so the total number of comparisons is $f(n)=n+\frac{n^2}{2}+\frac{3n}{2}-2=\frac{n^2}{2}+\frac{5n}{2}-2$. Since it is a polynomial of degree $2$, $f(n)=O(n^2)$.
\item We define a function $f:\mathbb{N}\longrightarrow\mathbb{N}$ that on input $n$ counts the worst-case number of comparisons (excluded comparisons in while loop and for loop) needed sort a list of length $n$. For a specific $k^{th}$ element to sort, the worst case for the while-loop is $2^q=k-1$, $q=\log_2(k-1)$. Because there are $N$ elements in the list, the complexity is $O(n\log_2 n)$. 
\par Since we know for insertion sort, the corresponding complexity is $O(n^2)$. Since $O(n\log_2 n)$ is way smaller than $O(n^2)$, the Binary Insertion Sort Algorithm is significantly faster than Insertion Sort.

\end{enumerate}











%========================================================================
\end{document}