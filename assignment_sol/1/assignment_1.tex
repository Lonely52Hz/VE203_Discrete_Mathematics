\newcommand{\plogo}{\fbox{$\mathcal{PL}$}} % Generic dummy publisher logo

%\usepackage[utf8]{inputenc} % Required for inputting international characters
%\usepackage[T1]{fontenc} % Output font encoding for international characters
%\usepackage{fouriernc} % Use the New Century Schoolbook font
\documentclass{article}[12pt]
\usepackage[margin=2.5cm]{geometry}
\usepackage{enumerate}
\usepackage{booktabs}
\usepackage{amsmath}
\newtheorem{theorem}{Theorem}  
\newtheorem{lemma}{Lemma}  
\newtheorem{proof}{Proof}
\usepackage{caption}
\usepackage{amssymb}
\usepackage{ulem}
\usepackage{graphicx}
\usepackage{subfigure}
\usepackage{geometry}
\usepackage{multirow}
\usepackage{multicol}
\usepackage{indentfirst}
\usepackage{xcolor}
\usepackage{verbatim}
%\usepackage{ctex}
\usepackage{gauss}
\usepackage{float}
\usepackage[version=4]{mhchem}

\begin{document}
\noindent

%========================================================================
\noindent\framebox[\linewidth]{\shortstack[c]{
\Large{\textbf{VE203 Assignment 1}}}}
\begin{center}
\footnotesize{\quad Name: YIN Guoxin\quad Student ID: 517370910043}


\end{center}
%=======================================================================

\noindent \textbf{Q1.}
\begin{enumerate}
\item Prove $\neg (A\wedge B) \Longleftrightarrow (\neg A \vee \neg B)$ using truth table.
\begin{table}[H]
\centering
\begin{tabular}{c|c|||c||c||c|c||c|||c}
$A$ & $B$ & $A\wedge B$ & $\neg (A\wedge B)$ & $\neg A $& $\neg B $&$\neg A \vee \neg B $&$ \neg (A\wedge B) \Longleftrightarrow (\neg A \vee \neg B) $\\ \hline
T & T & T                        & F                                              & F                     & F                     & F                                                               & T                                                                                                                                               \\
T & F & F                        & T                                              & F                     & T                     & T                                                               & T                                                                                                                                               \\
F & T & F                        & T                                              & T                     & F                     & T                                                               & T                                                                                                                                               \\
F & F & F                        & T                                              & T                     & T                     & T                                                               & T                                                                                                                                              
                                                                                                                                                
\end{tabular}
\end{table}
\item Prove $\neg (A\vee B) \Longleftrightarrow (\neg A \wedge \neg B)$ using truth table.
\begin{table}[H]
\centering
\begin{tabular}{c|c|||c||c||c|c||c|||c}
$A$ & $B$ & $A\vee B$ & $\neg (A\vee B)$ & $\neg A $& $\neg B $&$\neg A \wedge \neg B $&$ \neg (A\vee B) \Longleftrightarrow (\neg A \wedge \neg B) $\\ \hline
T & T & T                        & F                                              & F                     & F                     & F                                                               & T                                                                                                                                               \\
T & F & T                        & F                                              & F                     & T                     & F                                                               & T                                                                                                                                               \\
F & T & T                        & F                                              & T                     & F                     & F                                                               & T                                                                                                                                               \\
F & F & F                        & T                                              & T                     & T                     & T                                                               & T                                                                                                                                              

\end{tabular}
\end{table}
\end{enumerate}



\noindent \textbf{Q2.}
\begin{enumerate}[(i)]
\item Prove $M \backslash(A \cap B)=(M \backslash A) \cup(M \backslash B)$.
%\begin{itemize}
%\item First we prove $ M \backslash(A \cap B)\Rightarrow (M \backslash A) \cup(M \backslash B)$. Suppose $x\in M \backslash(A \cap B)$.
\begin{align*}
&x\in M \backslash(A \cap B) \Longleftrightarrow (x\in M)\wedge (x \not\in (A \cap B))\\\Longleftrightarrow &(x\in M)\wedge (\neg(x \in  (A \cap B)) 
\Longleftrightarrow (x\in M)\wedge (\neg(x \in A \wedge x\in B))\\
 \Longleftrightarrow &(x\in M)\wedge ((\neg( x\in A))\vee (\neg (x\in B))) \rm{\ (de\ Morgan\ rules)}
 \\
\Longleftrightarrow & (x\in M)\wedge (( x\not\in A)\vee (x\not\in B))\\
\Longleftrightarrow &((x\in M)\wedge (x\not\in A))\vee((x\in M)\wedge (x\not\in B)) \rm{\ (distributity)}\\
\Longleftrightarrow &x\in (M \backslash A) \cup(M \backslash B)
\end{align*}
%\item Second we prove $ (M \backslash A) \cup(M \backslash B)\Rightarrow M \backslash(A \cap B)$. Suppose $x\in (M \backslash A) \cup(M \backslash B)$.
%\end{itemize}

\item Prove $M \backslash(A \cup B)=(M \backslash A) \cap(M \backslash B)$.
\begin{align*}
&x\in M \backslash(A \cup B) \Longleftrightarrow (x\in M)\wedge (x \not\in (A \cup B))\\\Longleftrightarrow &(x\in M)\vee (\neg(x \in  (A \cup B)) 
\Longleftrightarrow (x\in M)\wedge (\neg(x \in A \vee x\in B)) \\
\Longleftrightarrow &(x\in M)\wedge ((\neg( x\in A))\wedge (\neg (x\in B)))\rm{\ (de\ Morgan\ rules)}
\\
\Longleftrightarrow & (x\in M)\wedge (( x\not\in A)\wedge (x\not\in B))\\ \Longleftrightarrow &(x\in M)\wedge(x\in M)\wedge ( x\not\in A)\wedge (x\not\in B)\rm{\ (Idempotecy \ of \ \wedge)}\\
\Longleftrightarrow &((x\in M)\wedge (x\not\in A))\wedge((x\in M)\wedge (x\not\in B))\rm{\ (commutativity)}\\
\Longleftrightarrow &x\in (M \backslash A) \cap(M \backslash B)
\end{align*}
\end{enumerate}

\noindent \textbf{Q3.}
The (i) and (iii) compound proposition are tautology. The reasons are listed below.
\begin{enumerate}[(i)]
\item 
\begin{align*}
&(A \Rightarrow(B \Rightarrow C)) \Rightarrow(B \Rightarrow(A \Rightarrow C))\\
\equiv &\neg  (A \Rightarrow(\neg B \vee C)) \vee (\neg B \vee(\neg A \vee C))\\
\equiv&\neg (\neg A \vee (\neg B \vee C) \vee (\neg B \vee (\neg A \vee C))\\
\equiv & \neg (\neg A \vee \neg B \vee C) \vee (\neg A \vee \neg B \vee C))\\
\equiv & (\neg A \vee \neg B \vee C)\Rightarrow (\neg A \vee \neg B \vee C)\\
\equiv &T
\end{align*}
Therefore, the (i) compound proposition is tautology.
\item 
\begin{align*}
&((A \vee B) \wedge(A \vee C)) \Rightarrow(B \vee C)\\
\equiv &\neg (A \vee (B\wedge C))\vee (B\vee C)
\equiv (\neg A \wedge(\neg (B \wedge C)) \vee (B\vee C)\\
\equiv &(\neg A \wedge(\neg B \vee \neg C)) \vee (B\vee C)
\equiv (B\vee C \vee (\neg A))\wedge (B\vee C \vee (\neg B) \vee (\neg C))\\
\equiv &B\vee C \vee (\neg A),
\end{align*}
which is not always equal to T. Therefore, it is not tautology.
\item 
\begin{align*}
&(A \Rightarrow(\neg B)) \Rightarrow(B \Rightarrow(\neg A))\\
\equiv &\neg (\neg A \vee \neg B) \vee (\neg B \vee \neg A)\\
\equiv & \neg (\neg A \vee \neg B) \vee (\neg A \vee \neg B)\\
\equiv &T
\end{align*}
Therefore, the (iii) compound proposition is tautology.
\end{enumerate}
\par
\noindent \textbf{Q4.} From the truth table, we can find how the combinations of truth or falsehood of each variable can result in a true proposition. For example, if there are three variables, namely $A$, $B$, $C$, and when $A$ is true, $B$ is true, $C$ is false, the proposition is true. And surely, other this kind of combinations may also be found in the truth table. For each combination, because we should satisfy the truth or falsehood of each variable at the same time, we use \textbf{conjunction} to connect each variable (if it's true in the combination) or its negation (if it's false in the combination). In the above example, it is $A\wedge B \wedge
\neg C$. Since there may exist other combinations that can result in a true proposition, we use \textbf{disjunction} to connect those combinations, which forms the resulting compound proposition. And therefore as long as one combination in the final compound proposition is satisfied, we can get a true value of the final proposition, otherwise we cannot.
\\
\\
\noindent \textbf{Q5.}
\begin{align*}
A\wedge B\equiv (\neg(\neg A)) \wedge (\neg (\neg B))\equiv \neg (\neg A \vee \neg B)
\end{align*}
\begin{align*}
A \Rightarrow B \equiv \neg A \vee B
\end{align*}
\begin{align*}
A \Leftrightarrow B &\equiv (A \wedge B)\vee ((\neg A)\wedge (\neg B))\\
&\equiv (\neg (\neg A \vee \neg B)) \vee (\neg (A \vee  B))
\end{align*}

\noindent \textbf{Q6.}
\begin{enumerate}[(i)]
\item 
%\begin{align*}
%&x\in ((X \cup Y) \backslash(X \cap Y)) \Longleftrightarrow (x\in (X \cup Y))\wedge (x\not\in (X \cap Y))\Longleftrightarrow (x\in (X \cup Y)) \wedge (\neg (x\in (X \cap Y)))\\
%\Longleftrightarrow & (x\in (X \cup Y)) \wedge (\neg (x\in X \wedge x\in Y))\Longleftrightarrow ((x\in X) \vee (x\in Y)) \wedge ((\neg(x\in X))\vee(\neg(x\in Y)))\\
%\Longleftrightarrow &(((x\in X )\vee (x\in Y))\wedge (\neg(x\in X)))\vee (((x\in X) \vee (x\in Y))\wedge (\neg(x\in Y)))\\
%\Longleftrightarrow & (F \vee ((x\in Y)\wedge (\neg(x\in X))))\vee (F \vee ((x\in X)\wedge (\neg(x\in Y))))\\
%\Longleftrightarrow & ((x\in Y)\wedge (x\not\in X))\vee ((x\in X)\wedge (x\not\in Y))\Longleftrightarrow  x\in((Y \backslash X) \cup(X \backslash Y))\\
%\Longleftrightarrow & x\in ((X \backslash Y) \cup(Y \backslash X))
%\end{align*}
\begin{align*}
&X \triangle Y
=(X \cup Y) \backslash(X \cap Y)\\=
&((X \cup Y)\backslash X) \cup ((X \cup Y)\backslash Y)\\=&((X \backslash X)\cup (Y \backslash X))\cup ((X \backslash Y)\cup (Y \backslash Y))\\
=&\emptyset\cup (Y \backslash X) \cup (X \backslash Y) \cup \emptyset\\
=&(X \backslash Y) \cup(Y \backslash X)
\end{align*}
\item Since $X,Y\subseteq M$, $M\backslash X=X^c$ and $M\backslash Y=Y^c$, then
\begin{align*}
&(M \backslash X) \triangle(M \backslash Y)
= X^c \triangle Y^c =(X^c \backslash Y^c) \cup(Y^c \backslash X^c)\\
=& (X^c \cap Y)\cup (Y^c \cap X)=(Y\backslash X)\cup (X\backslash Y)\\
=&(X \backslash Y) \cup(Y \backslash X)\\
=&X \triangle Y
\end{align*}
%To show $
%(M \backslash X) \triangle(M \backslash Y)\Longleftrightarrow((M \backslash X) \backslash (M \backslash Y)) \cup((M \backslash Y) \backslash (M \backslash X))\Longleftrightarrow X \triangle Y \Longleftrightarrow (X \backslash Y) \cup(Y \backslash X)$, we first prove that $((M \backslash X) \backslash (M \backslash Y))\Longleftrightarrow (X \backslash Y)$. Due the symmetry of $X$ and $Y$, the rest can be proved using the same way. Therefore, we can prove the equality. 
%\begin{align*}
%&x\in ((M \backslash X) \backslash (M \backslash Y))\Longleftrightarrow (x\in (M \backslash X) )\wedge (x\not\in (M \backslash Y))\\
%\Longleftrightarrow & ((x\in M)\wedge (x\not\in X))\wedge (\neg(x\in(M \backslash Y)))\\
%\Longleftrightarrow & ((x\in M)\wedge (\neg(x\in X)))\wedge (\neg((x\in M)\wedge (x\not\in Y)))\\
%\Longleftrightarrow & ((x\in M)\wedge (\neg(x\in X)))\wedge ((\neg (x\in M))\vee (\neg (x\in Y)))\\
%\Longleftrightarrow & ((x\in M)\wedge (\neg(x\in X))\wedge (\neg (x\in M)))\vee 
%((x\in M)\wedge (\neg(x\in X))\wedge (\neg (x\in Y)))\\
%\Longleftrightarrow & (F\wedge (\neg(x\in X)))\vee ((x\in M)\wedge (\neg(x\in X))\wedge (\neg (x\in Y)))\\
%\Longleftrightarrow & F \vee ((x\in M)\wedge (\neg(x\in X))\wedge (\neg (x\in Y))) \Longleftrightarrow (x\in M)\wedge (\neg(x\in X))\wedge (\neg (x\in Y))
%\end{align*}
\item 
\begin{align*}
&(X \triangle Y) \triangle Z= ((X \backslash Y) \cup(Y \backslash X))\triangle Z=((X \cap Y^c) \cup(Y \cap X^c))\triangle Z\\
=& (((X \cap Y^c) \cup(Y \cap X^c))\cap Z^c) \cup (Z\cap((X \cap Y^c) \cup(Y \cap X^c))^c)\\
=& (X \cap Y^c\cap Z^c)\cup (Y \cap X^c\cap Z^c)\cup (Z\cap ((X \cap Y^c)^c\cap (Y \cap X^c)^c))\\
=& (X \cap Y^c\cap Z^c)\cup (Y \cap X^c\cap Z^c)\cup (Z\cap ((X^c \cup Y)\cap (Y^c \cup X)))\\
=& (X \cap Y^c\cap Z^c)\cup (Y \cap X^c\cap Z^c)\cup (Z\cap (((X^c \cup Y)\cap Y^c) \cup ((X^c \cup Y)\cap X)))\\
=& (X \cap Y^c\cap Z^c)\cup (Y \cap X^c\cap Z^c)\cup (Z\cap ((X^c \cap Y^c)\cup (Y\cap Y^c)\cup (X^c\cap X) \cup (Y\cap X))\\
=& (X \cap Y^c\cap Z^c)\cup (Y \cap X^c\cap Z^c)\cup (Z\cap ((X^c \cap Y^c)\cup \emptyset\cup \emptyset \cup (Y\cap X))\\
=& (X \cap Y^c\cap Z^c)\cup (Y \cap X^c\cap Z^c)\cup (Z\cap X^c \cap Y^c) \cup (Z\cap Y\cap X)
\end{align*}
Due the symmetry of $X, Y, Z$ and $Y\triangle X=X\triangle Y$, which can be easily found from question 6(i),
\begin{align*}
&X \triangle (Y \triangle Z)=(Z \triangle Y)\triangle X\\
=&(Z \cap Y^c\cap X^c)\cup (Y \cap Z^c\cap X^c)\cup (X\cap Z^c \cap Y^c) \cup (X\cap Y\cap Z)
\end{align*}
Because of the commutativity of $\cup$, we see that the above two equations is equal. Therefore, we prove that the symmetric difference is associative.
\item \begin{align*}
&(X \cap Y) \triangle(X \cap Z)\\
=& ((X\cap Y)\cap (X\cap Z)^c)\cup ((X\cap Z)\cap (X\cap Y)^c)\\
=& ((X\cap Y)\cap (X^c \cup Z^c))\cup ((X\cap Z)\cap (X^c\cup Y^c))\\
=& ((X\cap Y\cap X^c) \cup (X\cap Y\cap Z^c))\cup ((X\cap Z\cap X^c)\cup (X\cap Z\cap  Y^c))\\
=& \emptyset\cup (X\cap Y\cap Z^c)\cup \emptyset \cup(X\cap Z\cap  Y^c)\\
=& X\cap ((Y\cap Z^c)\cup ( Z\cap  Y^c))\\
=& X \cap(Y \triangle Z)
\end{align*}
\end{enumerate}
\noindent \textbf{Q7.}
\begin{enumerate}[(i)]
\item 
\begin{align*}
&x\in X\triangle Y \Longleftrightarrow x\in ((X \backslash Y) \cup(Y \backslash X))\\
\Longleftrightarrow & (x\in (X \backslash Y))\vee (x\in (Y \backslash X))\\\Longleftrightarrow & ((x\in X)\wedge (x\not\in Y))\vee ((x\in Y)\wedge (x\not\in X))\\
\Longleftrightarrow & (A(x)\wedge \neg B(x))\vee (B(x)\wedge \neg A(x))
\end{align*}
Using a truth table below ($A$ represents $A(x)$ and $B$ represents $B(x)$), 
\begin{table}[H]\centering
\begin{tabular}{c|c||c|c||c|c||c||c|||c}
%$A(x)$ & $B(x)$ & $\neg A(x)$ & $\neg B(x)$ & $A(x)\wedge \neg B(x)$ & $B(x)\wedge \neg A(x)$ & $(A(x)\wedge \neg B(x))\vee (B(x)\wedge \neg A(x))$ & $A \oplus B$ & $((A(x)\wedge \neg B(x))\vee (B(x)\wedge \neg A(x)))\Longleftrightarrow (A\oplus B)$ \\
$A$ & $B$ & $\neg A$ & $\neg B$ & $A\wedge \neg B$ & $B\wedge \neg A$ & $(A\wedge \neg B)\vee (B\wedge \neg A)$ & $A \oplus B$ & $((A\wedge \neg B)\vee (B\wedge \neg A))\Leftrightarrow (A\oplus B)$ \\ \hline
T      & T      & F           & F           & F                      & F                      & F                                                   & F            & T                                                                                    \\
T      & F      & F           & T           & T                      & F                      & T                                                   & T            & T                                                                                    \\
F      & T      & T           & F           & F                      & T                      & T                                                   & T            & T                                                                                    \\
F      & F      & T           & T           & F                      & F                      & F                                                   & F            & T                                                                                   
\end{tabular}
\end{table}
\item Continue from last question (7.i),
\begin{align*}
&A\oplus B \Longleftrightarrow (A\wedge \neg B)\vee (B\wedge \neg A)\\
\Longleftrightarrow & (\neg (\neg  (A\wedge \neg B)))\vee (\neg (\neg  (B\wedge \neg A)))\\
\Longleftrightarrow & \neg ((\neg  (A\wedge \neg B))\wedge  (\neg  (B\wedge \neg A)))
\end{align*}
\item To check this is a valid argument, we need to verify that $((A\oplus B)\wedge (B\oplus C))\Longrightarrow (A\oplus C))$  is a tautology, which is shown in the truth table below.
\begin{table}[H]\centering
\begin{tabular}{c|c|c||c|c||c||c|c|||c}
$A$ & $B$ & $C$ & $A \oplus B$ & $B \oplus C$ & $(A \oplus B) \wedge (B \oplus C)$ & $A \oplus C$ & $\neg (A \oplus C)$ & $(A \oplus B) \wedge (B \oplus C)\Longrightarrow A \oplus C$ \\ \hline
T   & T   & T   & F            & F            & F                                  & F            & T                   & T                                                           \\
T   & T   & F   & F            & T            & F                                  & T            & F                   & T                                                           \\
T   & F   & T   & T            & T            & T                                  & F            & T                   & T                                                           \\
T   & F   & F   & T            & F            & F                                  & T            & F                   & T                                                           \\
F   & T   & T   & T            & F            & F                                  & T            & F                   & T                                                           \\
F   & T   & F   & T            & T            & T                                  & F            & T                   & T                                                           \\
F   & F   & T   & F            & T            & F                                  & T            & F                   & T                                                           \\
F   & F   & F   & F            & F            & F                                  & F            & T                   & T                                                          
\end{tabular}
\end{table}

\end{enumerate}

\noindent \textbf{Q8.}
To show $(\exists x(P(x) \Rightarrow Q(x))) \Longleftrightarrow((\forall x P(x)) \Rightarrow(\exists x Q(x)))$ is a tautology, we need to show that both $(\exists x(P(x) \Rightarrow Q(x))) \Longrightarrow((\forall x P(x)) \Rightarrow(\exists x Q(x)))$ and $(\exists x(P(x) \Rightarrow Q(x))) \Longleftarrow((\forall x P(x)) \Rightarrow(\exists x Q(x)))$ are tautology. To show the first expression is a tautology, we only need to prove that if $\exists x(P(x) \Rightarrow Q(x))$ is true, $(\forall x P(x)) \Rightarrow(\exists x Q(x))$ must be true, since if the former one is false, the implication is always true. Similarly, for the second expression, we only need to prove that if $(\forall x P(x)) \Rightarrow(\exists x Q(x))$  is true, $\exists x(P(x) \Rightarrow Q(x))$ must be true. The proof for these two cases are shown below.
\begin{enumerate}
\item 
We suppose $\exists x(P(x) \Rightarrow Q(x))$ is true.
Then $P(a) \Rightarrow Q(a)$, where $a$ is some (unknown) element of the domain of discourse, by \textit{Existential Instantiation}. There are two cases, 

\begin{itemize}
\item $P(a)$ holds, and then $Q(a)$ must holds.  
\begin{enumerate}
\item In this case, if we have $P(x)$ holds for all $x$ in the discourse, because we already know that $Q(a)$ also holds, $ \exists x Q(x) $ is also true. So $(\forall x P(x)) \Rightarrow(\exists x Q(x))$ must be true.
\item If there exists element in the discourse such that $P(x)$ is false, then $\forall x P(x)$ is false, then the implication must always be true because the antecedent is false. 
\end{enumerate}
\item $P(a)$ is false, and then $Q(a)$ can be either true or false. 
 Because $P(a)$ is false, then $\forall x P(x)$ is false, then the implication must always be true because the antecedent is false. 
\end{itemize}


\item We suppose $(\forall x P(x)) \Rightarrow(\exists x Q(x))$ is true. Then there are two cases,
\begin{itemize}
\item
\begin{enumerate}
\item $P(x)$ holds for any $x$ in the domain of the discourse, and there exists certain (unknown) $x_0$ such that $P(x_0)$ holds.
\item Therefore, we can find $a$ in the domain of the discourse, such that both $P(a)$ and $Q(a)$ holds. 
\item Therefore, $P(a) \Rightarrow Q(a)$.
\end{enumerate}
\item 
\begin{enumerate}
\item $P(a) $ doesn't hold, where $a$ is some (unknown) element of the domain of the discourse. 
\item So for $a$, whether $Q(a)$ is true or not, $P(a) \Rightarrow Q(a)$ always holds. 
\end{enumerate}
\end{itemize}
\end{enumerate}


\noindent \textbf{Q9.}
Since $T=(A \cap B) \cup((M \backslash A) \cap(M \backslash B))$, it only relates to set that is the subset of $M$. Therefore, $T\subseteq M$.
\begin{enumerate}
\item We let $x\in (M \backslash A) \cap B$, so $ x\in M $, $x\not\in A$ and $x\in B$. Since $B\subseteq M$, through simplifying, we get $x\not\in A$ and $x\in B$.
\item Since $x\in B$, $x\not\in (M \backslash B)$. Therefore, $x\not\in ((M \backslash A) \cap(M \backslash B))$.
\item Since $x\not\in A$, $x\not\in (A \cap B)$.
\item Due to 2 and 3, we conclude that $x\not\in T$.
\item Since $T \subseteq M$, we get $x\in M\backslash T$.
\end{enumerate}



\noindent \textbf{Q10.}

\begin{enumerate}[(i)]
\item The truth tables for $A\ |\ B$ and $A \downarrow B$ are listed below.
\begin{table}[H]
\centering
\begin{tabular}{c|c||c|c||c}
$A$ & $B$ & $A \wedge B$ & $\neg (A \wedge B)$ & $A\ |\ B$ \\ \hline
T   & T   & T            & F                   & F     \\
T   & F   & F            & T                   & T     \\
F   & T   & F            & T                   & T     \\
F   & F   & F            & T                   & T    
\end{tabular}
\end{table}

\begin{table}[H]
\centering
\begin{tabular}{c|c||c|c||c}
$A$ & $B$ & $A \vee B$ & $\neg (A \vee B)$ & $A \downarrow B$ \\ \hline
T   & T   & T          & F                   & F     \\
T   & F   & T          & F                   & F     \\
F   & T   & T          & F                   & F     \\
F   & F   & F          & T                   & T    
\end{tabular}
\end{table}
\item We can show that 
\begin{align*}
A\ |\ A \equiv \neg (A\wedge A)\equiv \neg A,
\end{align*}
Therefore, it can define the negation.
Besides, 
\begin{align*}
A\vee B &\equiv \neg (\neg A) \vee \neg (\neg B)\\
&\equiv \neg ((\neg A) \wedge (\neg B))\\
&\equiv (\neg A) \ |\ (\neg B)\\
&\equiv (A\ |\ A)\ |\ (B\ |\ B),
\end{align*}
Therefore, it can also define disjunction. From \textbf{Q4}, we know that we can only use three connectives ($\wedge$, $\vee$ and $\neg$) to form a compound proposition. And from \textbf{Q5}, we knows that the connectives $\vee$ and $\neg$ can be used to define $\wedge$. Therefore, every connective of propositional logic can be defined using only the connective $|$.

\item We can show that 
\begin{align*}
A\downarrow A \equiv \neg (A\vee A)\equiv \neg A,
\end{align*}
Therefore, it can define the negation.
Besides, 
\begin{align*}
A\vee B &\equiv \neg (A\downarrow B)\\
&\equiv (A\downarrow B) \downarrow (A\downarrow B)
\end{align*}
Therefore, it can also define disjunction. Due to the same reason with \textbf{Q10} (ii),  every connective of propositional logic can be defined using only the connective $\downarrow$.
\item No, there are not logically equivalent. Because if $A$ is true, $B$ is true and $C$ is false, then $B \downarrow C $ is false and $A \downarrow B $ is false from the truth table. However, $A \downarrow(B \downarrow C)$ is then false but $(A \downarrow B) \downarrow C$ is then true. Therefore, they are not logically equivalent.
\item To check this, we need to verify that $(((\neg A) \downarrow B)\wedge (A\ |\ C))\Longrightarrow (B\downarrow C) $ is a tautology. We use the truth table below to verify it. 
\begin{table}[H]\centering
\begin{tabular}{c|c|c|||c|c||c||c|||c|||c}
$A$ & $B$ & $C$ & $\neg A$ & $\neg A \downarrow B$ & $A\ |\ C$ & $(\neg A \downarrow B)\wedge (A\ |\ C)$ & $B\downarrow C$ & $(\neg A \downarrow B)\wedge (A\ |\ C)\Longrightarrow$ $B\downarrow C$ \\ 
\hline
T   & T   & T   & F        & F                     & F         & F                                       & F               & T                                                                      \\
T   & T   & F   & F        & F                     & T         & F                                       & F               & T                                                                      \\
T   & F   & T   & F        & T                     & F         & F                                       & F               & T                                                                      \\
T   & F   & F   & F        & T                     & T         & T                                       & T               & T                                                                      \\
F   & T   & T   & T        & F                     & T         & F                                       & F               & T                                                                      \\
F   & T   & F   & T        & F                     & T         & F                                       & F               & T                                                                      \\
F   & F   & T   & T        & F                     & T         & F                                       & F               & T                                                                      \\
F   & F   & F   & T        & F                     & T         & F                                       & F               & T                                                                     
\end{tabular}
\end{table}
\end{enumerate}









%========================================================================
\end{document}
